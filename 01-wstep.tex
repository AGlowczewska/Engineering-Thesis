\chapter{Wstęp}
Mimo spadku stopnia przestępczości w ostatnich latach w Polsce i stosunkowo wysokiego poziomu bezpieczeństwa zawsze lepiej mieć przy sobie narzędzia, które pozwolą kontrolować to co dzieje się w naszym domu. Kiedy jesteśmy w pracy, na wakacjach czy też zostawiliśmy nasz dom bez opieki na dłuższy czas chcielibyśmy mieć możliwość podglądu tego co się w nim dzieje i na bieżąco monitorować sytuację. W przypadku włamań zarejestrowany materiał filmowy jest bardzo cenny nie tylko dla nas ale szczególnie dla policji, która jest w stanie odtworzyć zachowania i czasami zidentyfikować sprawcę.  Bardzo często słyszy się też o zatruciach tlenkiem węgla, który nazywany jest cichym zabójcą czy wybuchach gazu w mieszkaniu. W takich chwilach człowiek zdaje sobie sprawę z zagrożenia najczęściej kiedy jest już za późno i są małe szanse aby uratować innych lokatorów naszego domu np. po zatruciu CO. 
Zbudowany system kontroli bezpieczeństwa - The Guard stara się rozwiązać te wszystkie problemy. Naszym celem było stworzenie systemu umożliwiającego analizę danych z czujników pomiarowych, monitorowanie pomieszczeń, w których zamontowano nasz system a także nagrywanie materiału video w momencie wykrycia ruchu i przechowywaniu go bezpiecznie na zewnętrznym serwerze aby był dostępny dla nas w każdym momencie i nie uległ zniszczeniu. Zadaniem systemu jest także poinformowanie o każdym niebezpieczeństwie właściciela systemu. Priorytetem był prosty i intuicyjny program obsługi, który mógłby być użyty przez każdą osobę, na każdej z najbardziej popularnych platform. Zdecydowano się na aplikację internetową i dwie aplikacje mobilne napisane natywnie dla systemu iOS i Android. Ponadto uzgodniono, że rozwiązanie będzie oparte na niezależnych modułach, które będzie można później, w łatwy sposób, zmodyfikować.

W ramach pracy przygotowano projekt całego systemu, od urządzeń zbierających dane, przez system monitorujący i analizujący zebrane dane, po aplikacje klienckie. Następnie zaimplementowano zaprojektowane wcześniej aplikacje, złożono zestawy urządzeń składających się z Raspberry Pi 3 i czujników, oraz połączono wszystkie elementy w spójny system.
Ze względu, na cel pracy oraz wykorzystane technologie i usługi, zespół oparł swoją pracę o: dokumentację usług dostępną na stronach internetowych producentów, dokumentację narzędzi dołączoną do odpowiednich repozytoriów, dokumentację sprzętu.

W ramach pracy Mateusz Bartos zrealizował aplikację mobilną przeznaczoną na system Android. Ponadto przygotował maszynę wirtualną w ramach usług oferowanych przez Microsoft Azure. Zaproponował też wykorzystane usługi Google: Firebase Realtime Database, Firebase Storage oraz Firebase Authentication. Piotr Falkiewicz wykonał projekt serwera, obsługującego zapytania aplikacji mobilnych, w oparciu o protokół HTTP oraz jest odpowiedzialny za poprawne przetwarzanie obrazu dostarczanego z urządzeń do aplikacji przy wykorzystaniu modułu Nginx Rtmp. Wprowadził do projektu obsługę usługi Firebase Storage. Paweł Szudrowicz w ramach pracy przygotował urządzenia Raspberry Pi 3 do obsługi czujników i wykonał oprogramowanie pracujące na każdym z nich. Kolejnym zrealizowanym zadaniem był projekt i wykonanie aplikacji mobilnej na urządzenia iOS. W ramach pracy Aleksandra Główczewska zaprojektowała i wykonała aplikację internetową, z wykorzystaniem języka programowania Python i frameworka Django. Ponadto jest odpowiedzialna za wprowadzenie uwierzytelniania użytkowników.

