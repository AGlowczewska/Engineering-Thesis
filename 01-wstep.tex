\chapter{Wstęp}
Inspiracją niniejszego projektu jest chęć stworzenia niezależnego systemu monitoringu wraz z~nowoczesnymi mechanizmami powszechnie używanymi w projektach programistycznych.
Współczesne systemy kontroli bezpieczeństwa powinny nie tylko nagrywać obraz, ale także analizować go w~czasie rzeczywistym i~odpowiednio reagować na wykryte zmiany. Na podstawie danych z~kamer i~czujników system powinien podejmować decyzje o stanie bezpieczeństwa domu i~w~razie potrzeby alarmować użytkownika o wykrytych zagrożeniach.

\section{Cel i zakres pracy}
System kontroli bezpieczeństwa - The Guard to nasza odpowiedź na przedstawione problemy. Celem pracy było stworzenie systemu umożliwiającego analizę danych z~czujników pomiarowych, monitorowanie pomieszczeń, w których zamontowano system, a~także nagrywanie materiału wideo w momencie wykrycia ruchu i~przechowywanie go bezpiecznie na zewnętrznym serwerze, aby był dostępny dla nas w każdym momencie i~nie uległ zniszczeniu. Do zaadań systemu należało także poinformowanie o~każdym niebezpieczeństwie właściciela systemu. Priorytetem był prosty i~intuicyjny program obsługi, który mógłby być użyty przez każdą osobę, na każdej z~najbardziej popularnych platform. Zdecydowano się na aplikację internetową oraz dwie aplikacje mobilne napisane natywnie dla systemu iOS i Android. Ponadto uzgodniono, że rozwiązanie będzie oparte na niezależnych modułach, które będzie można później w łatwy sposób zmodyfikować. Całość pracy oparta jest na licencji open-source, aby użytkownicy mogli nie tylko korzystać z~systemu, ale także dowolnie go edytować i~dopasowywać do własnych potrzeb.

W~ramach pracy przygotowano projekt całego systemu, od urządzeń zbierających dane, przez oprogramowanie monitorujące i~analizujące zebrane dane, po aplikacje klienckie. Następnie zaimplementowano zaprojektowane wcześniej aplikacje oraz złożono zestawy urządzeń składających się z~Raspberry Pi~3, czujników i~kamer.
Ze względu na cel pracy oraz wykorzystane technologie i~usługi, zespół oparł swoją pracę o dokumentację usług dostępną na stronach internetowych producentów, dokumentację narzędzi dołączoną do odpowiednich repozytoriów, dokumentację sprzętu.

\section{Podział pracy}
\begin{itemize}
\item Mateusz Bartos: \\
Zaprojektowanie architektury systemu, stworzenie aplikacji mobilnej przeznaczonej na system Android, przygotowanie maszyny wirtualnej w ramach usług oferowanych przez chmurę Microsoft Azure.
\item Piotr Falkiewicz: \\
Wykonanie projektu i zaimplementowanie serwera obsługującego aplikacje mobilne w oparciu o protokół HTTP, przetwarzanie strumienia obrazu dostarczanego w czasie rzeczywistym z~urządzeń do aplikacji przy wykorzystaniu modułu Nginx RTMP.
\item Aleksandra Główczewska: \\
Projekt i wykonanie aplikacji internetowej wraz z~własną obsługą baz danych, oparcie aplikacji na języku Python i bibliotece Django \cite{djangoREST}, wprowadzenie uwierzytelniania użytkowników.
\item Paweł Szudrowicz: \\
Przygotowanie urządzenia opartego na Raspberry Pi~3, wykonanie oprogramowania na Raspberry Pi~3 obsługującego czujniki i~kamerę w~czasie rzeczywistym. Ponadto, projekt i~wykonanie aplikacji mobilnej przeznaczonej na urządzenia z~systemem iOS, implementacja obsługi push notyfikacji wykorzystującej do tego usługę Firebase.
\end{itemize}

\section{Struktura pracy}
W~pierwszym rozdziale opisana jest architektura przygotowanego systemu. Następna część poświęcona jest kosztom utrzymania działającego systemu i~wszystkich wymaganych podzespołów do jego poprawnej pracy. Kolejny dział zawiera specyfikację wykorzystanych czujników i~schemat poprawnego ich podłączenia, a~także informacje dotyczące przetwarzania obrazu. W~części tej poruszona jest również kwestia poprawnej instalacji oprogramowania na urządzeniu Raspberry Pi~3.  W~czwartej części zaprezentowe są użyte rozwiązania chmurowe takie jak Microsoft Azure, Firebase i~omówione jest działanie serwera opartego na Django. Aplikacje klienckie są opisane w szóstym rozdziale poniższej pracy. W tym rozdziale wykorzystane są zdjęcia ekranów z~działających aplikacji wraz z~opisem najważniejszych aspektów ich realizacji. Ostatni rozdział zawiera opis przeprowadzonych testów funkcjonalności.
