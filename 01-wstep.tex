\begin{document}

\chapter{Wstęp}

Podjęcie tematu (WHY?)

Celem pracy było zaprojektowanie i wykonanie systeu umożliwiającego wspomniane wyżej funkcje. Ponadto uzgodniono, że rozwiązanie będzie oparte na niezależnych modułach, które będzie można później, w łatwy sposób, zmodyfikować.

W ramach pracy przygotowano propozycję projektu całego systemu, od urządzeń zbierających dane, przez system moniturujący, po aplikacje klienckie. Następnie zaimplementowano zaprojektowane aplikacje, złożono zestawy urządzeń Raspberry i czujników, oraz połączono wszystkie elementy w spójny system.

Ze względu, na cel pracy oraz wykorzystane technologie i usługi, zespół oparł swoją pracę o: dokumentację usług dostępną na stronach internetowych producentów, dokumentację narzędzi dołączoną do odpowiednich repozytoriów, dokumentację sprzętu.

Struktura pracy wygląda następująco: w 1. rozdziale opisano architekturę przygotowanego systemu. Następna częsc została poswięcona opisowi działaniu komputerów jednopłytkowych Raspberry Pi. W Rozdziale 3. opisane są wykorzystywane rozwiązania chmurowe. Rozdział 4. dotyczy aplikacji przeznaczonych dla klientów, a w rozdziale 5 zawarto podsumowanie pracy.

W ramach niniejszej pracy Mateusz Bartos wykonał projekt oraz zrealizował aplikację mobilną obsługiwaną przez system Android. Ponadto przygotował maszynę wirtualną w ramach usług oferowanych przez Microsoft Azure. Zaproponował też wykorzystane usługi Google: Firebase Realtime Database, Firebase Storage oraz Firebase Authentication. 
Piotr Falkiewicz wykonał i zrealizował projekt serwera, obsługującego zapytania aplikacji mobilnych, w oparciu o protokół HTTP oraz jest odpowiedzialny za właciwą obsługę  obrazu dostarczanego z urządzeń Raspberry do aplikacji przy wykorzystaniu modułu Nginx Rtmp. Ponadto wprowadził do projektu obsługę usługi Firebase Storage.
W ramach pracy, Aleksandra Główczewska zaprojektowała i wykonała aplikację internetową, z wykorzystaniem języka programowania Python i frameworka Django. Ponadto jest odpowiedzialna za wprowadzenie uwierzytelniania użytkowników, za pomocą Firebase Authentication. 
Paweł Szudrowicz, w ramach pracy, przygotował urządzenia Raspberry do obsługi czujników oraz wysyłania danych do usługi Firebase Realtime Databse. Kolejnym zadaniem, które zrealizował, był projekt i wykonanie aplikacji mobilnej na urządzenia iOS.
