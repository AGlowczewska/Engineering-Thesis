\chapter{Wstęp}
Inspiracją niniejszego projektu była chęć ulepszenia istniejących systemów monitoringu o nowoczesne mechanizmy powszechnie używane w projektach programistycznych.
Nowoczesne systemy kontroli bezpieczeństwa powinny nie tylko nagrywać obraz ale także analizować go w czasie rzeczywistym i odpowiednio reagować. Na podstawie danych z kamer i czujników system powinien podejmować decyzje o stanie bezpieczeństwa domu i alarmować użytkownika o wykrytych zagrożeniach.

System kontroli bezpieczeństwa - The Guard jest naszą odpowiedzią na przedstawione problemy. Naszym celem było stworzenie systemu umożliwiającego analizę danych z czujników pomiarowych, monitorowanie pomieszczeń, w których zamontowano nasz system a także nagrywanie materiału video w momencie wykrycia ruchu i przechowywaniu go bezpiecznie na zewnętrznym serwerze aby był dostępny dla nas w każdym momencie i nie uległ zniszczeniu. Zadaniem systemu jest także poinformowanie o każdym niebezpieczeństwie właściciela systemu. Priorytetem był prosty i intuicyjny program obsługi, który mógłby być użyty przez każdą osobę, na każdej z najbardziej popularnych platform. Zdecydowano się na aplikację internetową i dwie aplikacje mobilne napisane natywnie dla systemu iOS i Android. Ponadto uzgodniono, że rozwiązanie będzie oparte na niezależnych modułach, które będzie można później, w łatwy sposób, zmodyfikować. Całość pracy jest na licencji open-source, aby każdy użytkownik mógł nie tylko korzystać z systemu ale także dowolnie go edytować.

W ramach pracy przygotowano projekt całego systemu, od urządzeń zbierających dane, przez system monitorujący i analizujący zebrane dane, po aplikacje klienckie. Następnie zaimplementowano zaprojektowane wcześniej aplikacje, złożono zestawy urządzeń składających się z Raspberry Pi 3 i czujników, oraz połączono wszystkie elementy w jeden spójny system.
Ze względu na cel pracy oraz wykorzystane technologie i usługi, zespół oparł swoją pracę o: dokumentację usług dostępną na stronach internetowych producentów, dokumentację narzędzi dołączoną do odpowiednich repozytoriów, dokumentację sprzętu.

\paragraph{Podział pracy}
\begin{itemize}
\item Mateusz Bartos: \\
Zaprojektował architekturę systemu, stworzył aplikację mobilną przeznaczoną na system Android. Ponadto przygotował maszynę wirtualną w ramach usług oferowanych przez chmurę Microsoft Azure.
\item Piotr Falkiewicz: \\
Wykonał projekt serwera obsługującego aplikacje mobilne w oparciu o protokół HTTP oraz był odpowiedzialny za przetwarzanie monitoringu dostarczanego w czasie rzeczywistym z urządzeń do aplikacji przy wykorzystaniu modułu Nginx RTMP.
\item Aleksandra Główczewska: \\
Zaprojektowała i wykonała aplikację internetową z wykorzystaniem języka Python i biblioteki Django. Odpowiedzialna za wprowadzenie uwierzytelniania użytkowników.
\item Paweł Szudrowicz: \\
Przygotował urządzenia oparte na Raspberry Pi 3 do obsługi czujników i wykonał oprogramowanie pracujące na każdym z nich. Zaprojektował i wykonał aplikację mobilną przeznaczoną na urządzenia iOS a także zaimplementował obsługę push notyfikacji wykorzystując do tego usługi Firebase.
\end{itemize}

\paragraph{Struktura pracy} \\
W pierwszym rozdziale opisano architekturę przygotowanego systemu. Następna część poświęcona jest opisowi kosztów utrzymania działającego systemu i wszystkich wymaganych podzespołów do jego poprawnej pracy. Kolejny dział zawiera specyfikację wykorzystanych czujników i schemat poprawnego ich podłączenia a także informacje dot. przetwarzania obrazu. W części tej poruszono również kwestię poprawnej instalacji oprogramowania na urządzeniu Raspberry Pi 3.  W czwartej części zaprezentowano użyte rozwiązania chmurowe takie jak Microsoft Azure, Firebase i omówiono działanie naszego serwera opartego na Django. Aplikacje klienckie są kolejnym tematem tej pracy. Do ich opisania wykorzystano zdjęcia ekranów z działających aplikacji i wytłumaczono najważniejsze aspekty w ich realizacji. W ostatnim dziale opisano przeprowadzone testy funkcjonalne.