
\chapter{Parę słów o stylu \texttt{ppfcmthesis}}

\section{Różnice w stosunku do ,,oficjalnych'' zasad składu ze stron FCMu}

Autor niniejszego stylu nie zgadza się z niektórymi zasadami wprowadzonymi w oficjalnym 
dokumencie FCMu.\footnote{\url{http://www.fcm.put.poznan.pl/platon/dokumenty/dlaStudentow/egzaminDyplomowy/zasadyRedakcji}}
Poniższe elementy są składane nieco inaczej w stosunku do ,,oficjalnych'' wytycznych.

\begin{itemize}
    \item Promotor na stronie tytułowej jest umiejscowiony w centralnej osi pionowej strony (a
    nie po prawej stronie).
    
    \item Czcionka użyta do składu to nie Times New Roman.
    
    \item Spacje między tytułami akapitów oraz wcięcia zostały pozostawione takie, jak są zdefiniowane
    oryginalnie w pakiecie Memoir (oraz w \LaTeX{}u). Jeśli zdefiniowano ,,polską'' opcję składu,
    to będzie w użyciu wcięcie pierwszego akapitu po tytułach rozdziałów. Przy składzie ,,angielskim''
    tego wcięcia nie ma.

    \item Odwrócona jest kolejność rozdziałów \emph{Literatura} i \emph{Dodatki}.

    \item Na ostatniej stronie umieszczono stopkę informującą o prawach autorskich i programie
    użytym do składu.
    
    \item Nie do końca zgadzam się ze stwierdzeniem, iż ,,zamieszczanie list tabel, rysunków, 
    wykresów w pracy dyplomowej jest nieuzasadnione''. Niektóre typy publikacji zawierają tabele i rysunki, których
    skorowidz umożliwia łatwiejsze ich odszukanie. Ale niech będzie.

    \item Styl podpisów tabel jest taki sam, jak rysunków i odmienny od FCMowego. 
    Jeśli ktoś koniecznie chce mieć zgodne z wytycznymi
    podpisy, to zamiast \texttt{caption} niech użyje \texttt{fcmtcaption} do podpisywania tablic oraz
    \texttt{fcmfcaption} do podpisywania rysunków. Podpisy pod rysunkami pozostaną pełne, a nie skrócone (,,Rys.'').
    
    \item Styl formatowania literatury jest nieco inny niż proponowany przez FCM.
\end{itemize}

