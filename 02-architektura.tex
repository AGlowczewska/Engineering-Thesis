\chapter{Architektura systemu}

Wstęp do rozdziału

\section*{Schemat}

// Piotr - schemat na podstawie "Praca inżynierska na Google Docs"

\section*{Komunikacja}

// Piotr - opis requesteów i REST

\section*{Bezpieczeństwo}

Aplikacje wysyłając zapytania do serwera, muszą potwierdzić swoją tożsamość, co dzieje się inaczej w przypadku aplikacji mobilnych i aplikacji webowej.

W~przypadku aplikacji mobilnych zastosowano proponowane przez Firebase rozwiązanie JSON Web Tokens. W~momencie wysłania zapytania POST do serwera, aplikacja wysyła także unikalny token, który następnie jest przez serwer weryfikowany przy użyciu Firebase Admin SDK.

W~przypadku aplikacji internetowej, zastosowano wbudowane w bibliotekę Django zabezpieczenia: CSRF token oraz przesyłanie id sesji wraz z zapytaniem. Zabezpieczenie CSRF tokenem uniemożliwia tzw. `Cross Site Request Forgery' tj. ataki w~których na stronie, gdzie zalogowany jest użytkownik bez jego wiedzy uruchamiany jest skrypt, najczęściej w języku JavaScript. Następnie, korzystając z~faktu, że użytkownik jest zalogowany, wysyłane jest zapytanie na serwer, które może zrobić wszystkie operacje do których upoważniony jest dany użytkownik. CSRF token zapisywany jest w przeglądarce jako `ciasteczko' (eng. cookie) i jest dołączany do danych przesyłanych w momenie kliknięcia przycisku odpowiedzialnego za przesłanie formularza. Następnie wbudowana w serwer Django biblioteka weryfikuje na podstawie zapisanych i przesłanych danych sesji poprawność tokenu i w przypadku błędu zwraca błąd serwera 403.
Ponieważ token przy każdym zapytaniu jest tworzony na nowo na podstawie otwartej sesji, rozwiązanie nie było komfortowe dla użytkowników aplikacji mobilnych: aplikacja musiałaby najpierw ustanowić połączenie z serwerem (wysłać zapytanie GET na stronę główną), następnie zalogować się (wysłać zapytanie POST z danymi logowania) oraz zapisywać tokeny i id sesji odsyłane przez serwer. Aby ograniczyć ilość zapytań wysyłanych do serwera, posłużono się powyżej opisaną metodą tokenów JWT.
