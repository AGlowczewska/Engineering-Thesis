\chapter*{Streszczenie}
System kontroli bezpieczeństwa – The Guard służy do monitorowania zagrożeń w domu i~ostrzegania właścicieli przed niebezpieczeństwem. Pozwala na transmisję obrazu w czasie rzeczywistym za pośrednictwem chmury z pomieszczeń, w których zainstalowano urządzenia.
Zaprojektowano proste i~intuicyjne programy klienckie służace do obsługi systemu.
Zdecydowano się na aplikację internetową i dwie aplikacje mobile przeznaczone na system iOS i Android.
Dane zbierane są za pomocą urządzeń Raspberry Pi, wyposażonych w szereg czujników i kamery.
Jeżeli na którymkolwiek z czujników pojawią się odczyty odbiegające od normy, system powiadamia wszystkie urządzenia podłączone do konta użytkownika o zagrożeniu.
Za poprawne działanie samego systemu, odpowiadają usługi chmurowe Microsoft Azure oraz Google Firebase.
Cały system oparto na licencji open-source, co umożliwia użytkownikom wdrożenie systemu i modyfikację według własnych potrzeb.

{\let\clearpage\relax\chapter*{Summary}}


%\chapter{Summary}
Security control system The Guard is designed for real-time monitoring and notification in case of danger detection. It allows user to see the mounted cameras' and sensors' status via mobile and web applications. 
The designed applications are easy to use and do not require technical knowledge. Since it is important to have access to The Guard while on the go, the applications for most common platforms were created: Android, iOS and Web. The data are being collected and sent to server using Raspberry Pi~3 to which the camera and sensors are connected. When the data are different from standard values, the system informs user about the danger by sending notifications to every device connected to user's account. During implementation of the system, there were used such cloud services as Microsoft Azure and Google Firebase. The system is based on Open Source licence so that every user can adapt it to their needs.

