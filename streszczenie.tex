\chapter{Streszczenie}

System kontroli bezpieczeństwa – The Guard jest to system monitorowania zagrożeń w domu i ostrzegania właścicieli przed niebezpieczeństwem. Pozwala na transmisję obrazu w czasie rzeczywistym z pomieszczeń, w których zainstalowano urządzenia. Architekturę systemu oparto się na technologii przetwarzania w chmurze na takich platformach jak Microsoft Azure. Zaprojektowano prosty i intuicyjny program obsługi dostępny na każdej z najbardziej popularnych platform. Zdecydowano się na aplikację internetową i dwie aplikacje mobile przeznaczone na system iOS i Android. Każdy użytkownik systemu ma możliwość podłączenia dowolnej liczby zestawów pomiarowych. Pojedynczy zestaw składa się z kamery nagrywającej obraz w FULL HD, czujników LPG, CO i temperatury, a także z czujników wykrywania płomieni i ruchu. Jeżeli na jakimkolwiek z czujników pojawią się odczyty odbiegające od normy, system powiadamia wszystkie urządzenia podłączone do konta użytkownika o zagrożeniu. Powiadomienia realizowane są przy użyciu push notyfikacji. 
Podobne zachowanie zaobserwujemy w momencie wykrycia ruchu spowodowanego przez człowieka. Dodatkowo jednak system nagrywa 30 sekundowy materiał wideo dołączając jego adres URL do wysłanej notyfikacji. Analiza obrazu odbywa się na zewnętrznym serwerze wyposażonym w zaawansowane podzespoły dzięki czemu przebiega sprawnie i szybko. Wszystkie informacje o zagrożeniach zapisywane są w bazie danych. Użytkownik ma możliwość odtworzenia całej historii wydarzeń w systemie. Dane z czujników aktualizowane są nieprzerwanie co 2 sekundy zapewniając ciągłość pomiarów. Aplikacje klienckie odbierają wszystkie te dane w czasie rzeczywistym dzięki usłudze Firebase. Postanowiono, że cały system będzie na licencji open-source, aby każdy mógł pobrać oprogramowanie i edytować je według własnych potrzeb.




System kontroli bezpieczeństwa – The Guard jest to system monitorowania zagrożeń w domu i ostrzegania właścicieli przed niebezpieczeństwem. Pozwala on na kontrolę aktualnego stanu bezpieczeństwa, przy użyciu aplikacji na platformy Android, iOS oraz Web. Dane zbierane są za pomocą urządzeń Raspberry Pi, wyposażonych w szereg czujników i kamery. Za bezpieczeństwo i poprawne działanie samego systemu, odpowiadają usługi chmurowe MS Azure oraz Google Firebase. 
