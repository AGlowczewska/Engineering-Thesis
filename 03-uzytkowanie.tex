\chapter{Kosztorys}

\section{Koszty elementów systemu}
W tym rozdziale postarano się o umieszczenie wszystkich kosztów budowy systemu i kosztów jego utrzymania. Koszt budowy systemu zależy od ilości posiadanych urządzeń pomiarowych. Zaprezentowany zostanie koszt tylko jednego urządzenia, który wygląda następująco:
\begin{itemize}
\item Koszty czujników:
Do budowy użyto czujników: MQ-9 \cite{specyfikacjaMQ-9}, MQ-2 \cite{specyfikacjaMQ-2}, czujnik DS18B20+ \cite{specyfikacjaTemp}, czujnik ruchu \cite{pir}, czujnik wykrycia płomieni \cite {specyfikacjaFlame}. Na stronach odnośników znajdują się ich ceny ze sklepu Botland.pl. Sumaryczny koszt: 91,60 zł.
\item Koszt kamery:
Użyto kamery 5MP Full HD ze wsparciem do modułu Raspberry Pi 3. Jej koszt wynosi: 89,00zł.
\item Koszt przetwornika AC:
Użyto przetwornika MCP3008 \cite{specyfikacjaAC}. Jego koszt wynosi: 9,90zł.
\item Koszt Raspberry Pi 3 + karta pamięci 16GB: 
Ze względu na bardzo dużą ilość dostępnych źródeł, ceny jego nabycia różnią się znacząco. Cena z Botland.pl: 209,90zł
\end{itemize}
Całkowity koszt jednego urządzenia wynosi zatem około 400,40zł. Istnieje możliwość obniżenia tej kwoty poprzez zakup kamery nagrywającej w mniejszej rozdzielczości. 

\section{Koszty użytkowania}
Poza jednorazowymi kosztami związanymi z montażem elementów systemu użytkownik ponosi także koszty miesięczne związane z utrzymaniem części systemu znajdującej się w chmurze.
\begin{itemize}
    \item Koszty maszyny wirtualnej w chmurze Microsoft Azure \newline
    Przy wykorzystaniu maszyny klasy B2MS miesięczny koszt działania wynosi \textbf{60,98 €}
    \item Koszty usług Google Firebase\newline
    Usługa jest darmowa jeżeli przestrzegane są poniższe limity:
    \subitem \textbf{Firebase Realtime Database}
    \subsubitem - \textbf{100} równoczesnych połączeń,
    \subsubitem - \textbf{1 GB} przechowywanych danych,
    \subsubitem - \textbf{10 GB} danych transferowanych w ciągu miesiąca.
    \subitem \textbf{Firebase Storage}
    \subsubitem - \textbf{1 GB} danych pobieranych w ciągu dnia,
    \subsubitem - \textbf{5 GB} przechowywanych danych,
    \subsubitem - \textbf{20 000} operacji wysyłania danych dziennie,
    \subsubitem - \textbf{50 000} operacji pobierania danych dziennie.
\end{itemize}