\chapter{Kosztorys}

\section{Koszty elementów systemu}
W tym rozdziale postarano się o umieszczenie wszystkich kosztów budowy systemu i kosztów jego utrzymania. Koszt budowy systemu zależy od ilości posiadanych urządzeń pomiarowych. Zaprezentowany zostanie koszt tylko jednego urządzenia, który wygląda następująco:
\begin{enumerate}
\item Koszty czujników:
Do budowy użyto czujników: MQ-9 \cite{specyfikacjaMQ-9}, MQ-2 \cite{specyfikacjaMQ-2}, czujnik DS18B20+ \cite{specyfikacjaTemp}, czujnik ruchu \cite{pir}, czujnik wykrycia płomieni \cite {specyfikacjaFlame}. Na stronach odnośników znajdują się ich ceny ze sklepu Botland.pl. Sumaryczny koszt: 91,60 zł.
\item Koszt kamery:
Użyto kamery 5MP Full HD ze wsparciem do modułu Raspberry Pi 3. Jej koszt wynosi: 89,00zł.
\item Koszt przetwornika AC:
Użyto przetwornika MCP3008 \cite{specyfikacjaAC}. Jego koszt wynosi: 9,90zł.
\item Koszt Raspberry Pi 3 + karta pamięci 16GB: 
Ze względu na bardzo dużą ilość dostępnych źródeł, ceny jego nabycia różnią się znacząco. Cena z Botland.pl: 209,90zł
\end{enumerate}
Całkowity koszt jednego urządzenia wynosi zatem około 400,40zł. Istnieje możliwość obniżenia tej kwoty poprzez zakup kamery nagrywającej w mniejszej rozdzielczości. 

\section{Koszty użytkowania}

