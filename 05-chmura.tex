\chapter{Rozwiązania chmurowe}

\section{Microsoft Azure}

Aby zapewnić wysoki poziom bezpieczeństwa oraz dostępności systemu zdecydowano się na skorzystanie z chmury Micrsoft Azure.

\section{Firebase}

// Mateusz

\section{Aplikacja serwerowa}

\paragraph*{Koncepcja}
Aplikacja serwerowa, została zaplanowana jako interfejs pomiędzy: urządzeniami peryferyjnymi, aplikacjami użytkowników urządzeń mobilnych, bazą danych a usługami chmurowymi.
Założeniem, dotyczącym pierwsza częsci komunikacji: Raspberry - Serwer - Aplikacja, było wykorzystanie zapytań HTTP: GET i POST.
\begin{itemize}
\item Zapytanie GET 
\item Zapytanie POST
\end{itemize}

\paragraph*{final version}
W ostatecznej wersji, skorzystano z dodatkowych usług, m. in Firebase OAuth2. Uwierzytelnianie użytkowników zaczęło wymagać, by do każdego zapytania, dodać token użytkownika. Spowodowało to, że jedynym typem zapytań, wykorzystywanym w systemie, są zapytania POST. 
\cite{djangoREST}

// Ola

\section{Baza danych}

// Ola
