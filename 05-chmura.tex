\chapter{Rozwiązania chmurowe}

\section{Microsoft Azure}

Aby zapewnić wysoki poziom bezpieczeństwa oraz dostępności systemu zdecydowano się na skorzystanie z~chmury Micrsoft Azure.
Wykorzystany został serwer wirtualny, na którym zainstalowano system Ubuntu 16.04 LTS. Dostęp do serwera możliwy jest wyłącznie po protokole SSH.

\paragraph{Parametry maszyny wirtualnej 'Anton'}
\begin{itemize}
    \item \textbf{Klasa}: B2MS
    \item \textbf{Lokalizacja}: Zachodnia Europa
    \item \textbf{Moc obliczeniowa}: 2 vCPU
    \item \textbf{Pamięć RAM}: 8 GB
    \item \textbf{Dysk SSD}: 16 GB
    \item \textbf{Ilość dysków danych}: 4
    \item \textbf{Ilość operacji wejścia/wyjścia na sekundę}: 4800
    \item \textbf{Koszt}: 60,98 € miesięcznie
\end{itemize}

Po skonfigurowaniu środowisk deweloperskich zarówno dla części obsługującej aplikacje mobilne oraz części odpowiedzialnej za przetwarzanie obrazu, wybrany katalog podłączono do repozytorium kodu korzystającego z systemu kontroli wersji "git", które znajduje się w~serwisie GitHub.com.
W ten sposób bieżące zmiany były dokonywane na komputerach deweloperów, którzy za pomocą repozytorium publikowali nowe wersje aplikacji serwerowej w~chmurze.

\section{Firebase}

W~celu usprawnienia działania aplikacji klienckich skorzystano także z~usług platformy Firebase. Platforma Firebase jest częścią chmury Google Cloud i~oferuje między innymi:

\begin{itemize}
    \item \textbf{Firebase Auth} - usługę bezpiecznej autoryzacji użytkowników ,
    \item \textbf{Firebase Storage} - usługę wygodnego przechowywania plików,
    \item \textbf{Firebase Realtime Database} - usługę bazy danych aktualizowanej w czasie rzeczywistym.
\end{itemize}

Usługi autoryzacji użytkowników zostały wykorzystane nie tylko w aplikacjach klienckich, ale i~także na serwerze do autoryzacji tokenów z~nadchodzących żądań klientów. Usługa przechowywania plików została wykorzystywana do przesyłania fragmentów nagrań na których wykryto niebezpieczeństwo. Usługa bazy danych została wykorzystana do prezentowania w~czasie rzeczywistym wartości czujników w~aplikacjach klienckich.


\section{Aplikacja serwerowa}

Serwer, obsługujący aplikacje mobilne, wykonano we frameworku języka Python - Django, z~wykorzystaniem modułu Django-Rest-Framework. Umożliwił on tworzenie endpointów, które, zgodnie z~rozdziałem nt. komunikacji, które mogły odbierać zapytania POST.
Obsługę zapytań można podzielić, ze względu na zaplanowane źródło zapytania: aplikacja użytkownika lub urządzenie Raspberry.
W~pierwszej kolejności przedstawione zostaną wiadomości wymieniane na linii Raspberry - Serwer.
\paragraph{a) Rejestracja Raspberry Pi:}
\begin{verbatim}
Adres: /backend/v1/devices/add
Zawartość:
{
	'serial': <serial-urządzenia>, 
	'name': <nazwa-urządzenia>, 
	'token': 'jwt.token.from.client'
}
\end{verbatim}
Działanie: Raspberry, o~podanym numerze seryjnym i~nazwie, zostaje dodane do bazy danych urządzeń.

\paragraph{b) Wykrycie ruchu:}
\begin{verbatim}
Adres: /backend/v1/PIRnotification
Zawartość: 
{
	'serial': <serial-urządzenia>, 
	'message': <wiadomość>, 
	'token': 'jwt.token.from.client'
}
\end{verbatim}
Działanie: Po odebraniu informacji o~wykryciu ruchu, następuje pobranie klatki ze strumienia obrazu nadawanego przez Raspberry o~wskazanym numerze seryjnym. Jeżeli na pobranej klatce wykryto człowieka, uruchamiana jest funkcja nagrywająca 30 sekundowy fragment wideo, który zostaje zapisany w~bazie danych Firebase Storage. Użytkownik zostaje poinformowany o~zajściu zdarzenia z~informacją zawartą w polu 'message'. Notyfikacja zostanie wysłana jeżeli wykrycie ruchu zostało spowodowane przez człowieka. W~każdym innym przypadku zostanie zignorowana.

\paragraph{c) Wykrycie zmian na czujniku:}
\begin{verbatim}
Adres: /backend/v1/notification
Zawartość: 
{
	'serial': <serial-urządzenia>, 
	'sensorType': <typ-czujnika>, 
	'value': <wartość>, 
	'token': `jwt.token.from.client`
}
\end{verbatim}
Działanie: Informuje serwer o~wykryciu zagrożenia na jednym z~czujników. Serwer następnie wyszukuje wszystkich klientów, którzy posiadają urządzenie o~numerze seryjnym, który wykrył niebezpieczne wskazania na czujniku i~wysyła do nich powiadomienie za pomocą push notyfikacji. \newline
Następne zapytania dotyczą poleceń wysyłanych z~aplikacji użytkownika.
\paragraph{d) Pobranie urządzeń użytkownika:}
\begin{verbatim}
Adres: /backend/v1/get
Zawartość: 
{
	'owner':<użytkownik>, 
	'token': 'jwt.token.from.client'
}
\end{verbatim}
Działanie: Zwraca listę urządzeń użytkownika.
\paragraph{e) Zmiana nazwy urządzenia:}
\begin{verbatim}
Adres: /backend/v1/devices/changeRaspName
Zawartość:
{
	'serial': <serial-urządzenia>, 
	'name': <nowa-nazwa>, 
	'token': 'jwt.token.from.client'
}
\end{verbatim}
Działanie: Zmienia nazwę urządzenia, wyświetlaną w~aplikacji użytkownika. Przyjęto, że nazwa ta powinna oznaczać miejsce, w~którym znajduje się urządzenie.
\paragraph{f) Uzbrojenie/rozbrojenie urządzenia:}
\begin{verbatim}
Adres: /backend/v1/devices/changeIsArmed

Zawartość: 
{
	'serial': <serial-urządzenia>, 
	'armed': <nowy-stan>, 
	'token': 'jwt.token.from.client'
}
\end{verbatim}
Działanie: Ustala czy nowe powiadomienia związane z~urządzeniem dalej będą wysyłane do aplikacji.
\paragraph{g) Pobranie listy notyfikacji:}
\begin{verbatim}
Adres: /backend/v1/devices/getNotifications
Zawartość: 
{
	'serial': <serial-urządzenia>,  
	'token': 'jwt.token.from.client'
}
\end{verbatim}
Działanie: Pobiera listę notyfikacji (historię zdarzeń) powiązanych z~urządzeniem o~podanym numerze seryjnym.
\paragraph{h) Powiązanie aplikacji z kontem użytkownika:}
\begin{verbatim}
Adres: /backend/v1/devices/fcmTokenUpdate
Zawartość: 
{
	'email': <użytkownik>, 
	'fcmToken': <token-z-firebase>, 
	'deviceId' : <id_aplikacji>
}
\end{verbatim}
Działanie: Powiązuje aplikację mobilną jak i~sesję aplikacji przeglądarkowej o~podanym tokenie z~kontem użytkownika. Dzięki temu, notyfikacje trafiają na wszystkie urządzenia użytkownika.
\newline
\newline
Wybrane rozwiązanie pozwala na łatwą lokalizację ewentualnego błędu w działaniu systemu oraz szybką jego naprawę. Ponadto prosta logika oraz łatwe i~krótkie funkcje obsługujące zapytania sprawiają, że dalszy rozwój tej części systemu będzie możliwy bardzo niskim nakładem sił. 

\section{Baza danych}

// Ola
