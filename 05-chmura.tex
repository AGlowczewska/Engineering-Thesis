\chapter{Rozwiązania chmurowe}

\section{Microsoft Azure}

Aby zapewnić wysoki poziom bezpieczeństwa oraz dostępności systemu zdecydowano się na skorzystanie z chmury Micrsoft Azure.
Wykorzystany został serwer wirtualny, na którym zainstalowano system Ubuntu 16.04 LTS. Dostęp do serwera możliwy jest wyłącznie po protokole SSH.

\paragraph{Parametry maszyny wirtualnej 'Anton'}
\begin{itemize}
    \item \textbf{Klasa}: B2MS
    \item \textbf{Lokalizacja}: Zachodnia Europa
    \item \textbf{Moc obliczeniowa}: 2 vCPU
    \item \textbf{Pamięć RAM}: 8 GB
    \item \textbf{Dysk SSD}: 16 GB
    \item \textbf{Ilość dysków danych}: 4
    \item \textbf{Ilość operacji wejścia/wyjścia na sekundę}: 4800
\end{itemize}

Po skonfigurowaniu środowisk deweloperskich zarówno dla części obsługującej aplikacje mobilne oraz części odpowiedzialnej za przetwarzanie obrazu, wybrany katalog podłączono do repozytorium kodu korzystającego z systemu kontroli wersji  "git", które znajduje się w serwisie GitHub.com.
W ten sposób bieżące zmiany były dokonywane na komputerach deweloperów, którzy za pomocą repozytorium publikowali nowe wersje aplikacji serwerowej w chmurze.

\section{Firebase}

W celu usprawnienia działania aplikacji klienckich skorzystano także z usług platformy Firebase. Platforma Firebase jest częścią chmury Google Cloud i oferuje między innymi:

\begin{itemize}
    \item \textbf{Firebase Auth} - usługę bezpiecznej autoryzacji użytkowników ,
    \item \textbf{Firebase Storage} - usługę wygodnego przechowywania plików,
    \item \textbf{Firebase Realtime Database} - usługę bazy danych aktualizowanej w czasie rzeczywistym.
\end{itemize}

Usługi autoryzacji użytkowników zostały wykorzystane nie tylko w aplikacjach klienckich, ale i także na serwerze do autoryzacji tokenów z nadchodzących żądań klientów. Usługa przechowywania plików została wykorzystywana do przesyłania fragmentów nagrań na których wykryto niebezpieczeństwo. Usługa bazy danych została wykorzystana do prezentowania w czasie rzeczywistym wartości czujników w aplikacjach klienckich.

\section{Aplikacja serwerowa}

// Ola

\section{Baza danych}

// Ola
