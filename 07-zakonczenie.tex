\chapter{Testy funkcjonalne}

Testy:
 * Rejestracja+dodanie raspa
 * Aplikacji (3x, dla każdej):
 ** Sprawdzenie zagrożenia (skok wartości jednego z czujników)
 ** Sprawdzenie obrazu z kamery na wybranym raspie
 ** Odtworzenie nagrania video

\chapter{Uwagi końcowe}

Grupa zrealizowała wszystkie cele postawione sobie na początku pracy. Wykonany system działa i jest przystosowany do dalszej modyfikacji. Istnieje możliwość wymiany czujników z rodziny MQ na inne bez konieczności zmian w oprogramowaniu uruchomionym na Raspberry, ze względu na podobny charakter ich działania. W następnej fazie zalecane byłoby zaprojektowanie lepszej obudowy na czujniki aby poprawić design systemu. Zdecydowano, że cały system będzie na licencji open-source, aby każdy mógł pobrać oprogramowanie i edytować je według własnych potrzeb. Mimo, że taktowanie procesora Raspberry Pi 3 nie jest wysokie i nie poradziłby on sobie sam z zadaniami przedstawionymi na początku pracy to dzięki zastosowaniu zewnętrzych serwerów i usług na takich platformach jak Microstof Azure udało się wykonać w pełni działający system bezpieczeństwa. Przeniesiono bardzo obciążające zadania takie jak przetwarzanie obrazu na zewnętrzne platformy, które wyposażone są w znacznie bardziej zaawansowane podzespoły. Wykorzystanie natomiast usług Firebase, z którego korzystają też takie firmy jak Trivago czy Shazam pozwoliło na szybką implementację zaawansowanych funkcji takich jak uwierzytelnianie użytkowników czy aktualizację zmian w bazie danych w czasie rzeczywistym. Grupa zamierza dalej pracować nad systemem kontroli bezpieczeństwa - The Guard aby uczynić świat lepszym a przede wszystkim bezpieczniejszym miejscem do życia :)


// TODO Remove dummy doc links
\cite{MDESIGN}
\cite{RXJAVA}
\cite{KOTLIN}
\cite{RPI}
\cite{firebase}
\cite{android}
\cite{azure}
\cite{kotlin}