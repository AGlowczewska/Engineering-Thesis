\chapter{Testy funkcjonalne}

\paragraph{Test nr 1}
Zakres testu:\newline
Użytkownik po odebraniu notyfikacji w aplikacji przeglądarkowej ma skontrolować zagrożenie\newline
Tester: Piotr Falkiewicz\newline
Autor aplikacji: Aleksandra Główczewska\newline
Przebieg testu:\newline
1. Użytkownik otwiera aplikację.\newline
2. Tester odebrał informację o zagrożeniu.\newline
3. Po naciśnięciu przycisku 'Check It!' użytkownik został przeniesiony na podstronę związaną z notyfikacją.\newline
Wniosek:\newline
Funkcjonalność zaimplementowana prawidłowo\newline
\newline

\paragraph{Test nr 2}
Zakres testu:\newline
Użytkownik dodaje nowe urządzenie do swojego konta\newline
Tester: Aleksandra Główczewska \newline
Autor aplikacji: Paweł Szudrowicz \newline
Przebieg testu:\newline
1. Użytkownik otwiera aplikację.\newline
2. Użytkownik wybiera przycisk 'New' w dolnej części ekranu \newline
3. Użytkownik wprowadza numer seryjny urządzenia, które zamierza dodać \newline
4. Po naciśnięciu przycisku 'Send' użytkownik otrzymał informację o sukcesie.\newline
Wniosek:\newline
Funkcjonalność zaimplementowana prawidłowo\newline
\newline
 
 \paragraph{Test nr 3}
Zakres testu:\newline
Użytkownik ma otrzymać transmisję obrazu z urządzenia\newline
Tester: Piotr Falkiewicz \newline
Autor aplikacji: Paweł Szudrowicz \newline
Przebieg testu:\newline
1. Użytkownik otwiera aplikacje \newline
2. Użytkownik zaznacza urządzenie, z którego zamierza odbierać obraz \newline
3. Po chwili w górnej części ekranu pojawił się transmitowany obraz\newline
Wniosek:\newline
Funkcjonalność zaimplementowana prawidłowo\newline
\newline

 \paragraph{Test nr 4}
Zakres testu:\newline
Użytkownik ma pobrać nagranie z bazy danych z ostatniego zagrożenia\newline
Tester: Mateusz Bartos \newline
Autor aplikacji: Paweł Szudrowicz \newline
Przebieg testu:\newline
1. Użytkownik otwiera aplikacje \newline
2. Użytkownik zaznacza urządzenie, z którego zamierza pobrać materiał \newline
3. Użytkownik przechodzi do sekcji 'ostatnie zagrożenie' \newline
4. Użytkownik wybiera przycisk 'preview' \newline
5. Na ekranie pojawia się ekran ładowania \newline
6. Po chwili w górnej części ekranu pojawiło się pobrane nagranie \newline
Wniosek:\newline
Funkcjonalność zaimplementowana prawidłowo\newline


\chapter{Uwagi końcowe}

Grupa zrealizowała wszystkie cele postawione sobie na początku pracy. Wykonany system działa i jest przystosowany do dalszej modyfikacji. Istnieje możliwość wymiany czujników z rodziny MQ na inne bez konieczności zmian w oprogramowaniu uruchomionym na Raspberry, ze względu na podobny charakter ich działania. W następnej fazie zalecane byłoby zaprojektowanie nowej obudowy na czujniki aby poprawić wygląd urządzeń. Zdecydowano, że cały system będzie na licencji open-source, aby każdy mógł pobrać oprogramowanie i edytować je według własnych potrzeb. Mimo, że taktowanie procesora Raspberry Pi 3 nie jest wysokie i nie poradziłby on sobie sam z zadaniami przedstawionymi na początku pracy to dzięki zastosowaniu zewnętrzych serwerów i usług na takich platformach jak Microstof Azure udało się wykonać w pełni działający system bezpieczeństwa. Przeniesiono bardzo obciążające zadania takie jak przetwarzanie obrazu na zewnętrzne platformy, które wyposażone są w znacznie bardziej zaawansowane podzespoły. Wykorzystanie natomiast usług Firebase, z którego korzystają też takie firmy jak Trivago czy Shazam pozwoliło na szybką implementację zaawansowanych funkcji takich jak uwierzytelnianie użytkowników czy aktualizację zmian w bazie danych w czasie rzeczywistym. Grupa zamierza dalej pracować nad systemem kontroli bezpieczeństwa - The Guard aby uczynić świat lepszym a przede wszystkim bezpieczniejszym miejscem do życia :)
