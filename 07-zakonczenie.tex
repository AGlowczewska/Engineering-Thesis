\chapter{Testy funkcjonalne}


\paragraph{Test nr 1}
Zakres testu:\newline
Użytkownik, po odebraniu notyfikacji w aplikacji przeglądarkowej, ma skontrolować zagrożenie\newline
Tester: Piotr Falkiewicz\newline
Autor aplikacji: Aleksandra Główczewska\newline
Przebieg testu:\newline
1. Użytkownik otwiera aplikację.\newline
2. Tester odebrał informację o zagrożeniu.\newline
3. Po naciśnięciu przycisku 'Check It!' użytkownik został przeniesiony na podstronę związaną z notyfikacją.\newline
Wniosek:\newline
Funkcjonalnoć zaimplementowana prawidłowo\newline
\newline
//////////////////////\newline
\paragraph{Test nr 2}
Zakres testu:\newline
Użytkownik dodaje nowe urządzenie do swojego konta\newline
Tester: XXX \newline
Autor aplikacji: XXX \newline
Przebieg testu:\newline
1. Użytkownik wybiera przycisk 'Dodaj' w dolnej części ekranu \newline
2. Użytkownik wprowadza numer seryjny urządzenia, które zamierza dodać \newline
3. Po naciśnięciu przycisku 'Send' użytkownik otrzymał informację o sukcesie.\newline
Wniosek:\newline
Funkcjonalnoć zaimplementowana prawidłowo\newline

 
 
 
 * Aplikacji (3x, dla każdej):\newline
 ** Sprawdzenie zagrożenia (skok wartości jednego z czujników)\newline
 ** Sprawdzenie obrazu z kamery na wybranym raspie\newline
 ** Odtworzenie nagrania video\newline

\chapter{Uwagi końcowe}

Grupa zrealizowała wszystkie cele postawione sobie na początku pracy. Wykonany system działa i jest przystosowany do dalszej modyfikacji. Istnieje możliwość wymiany czujników z rodziny MQ na inne bez konieczności zmian w oprogramowaniu uruchomionym na Raspberry, ze względu na podobny charakter ich działania. W następnej fazie zalecane byłoby zaprojektowanie lepszej obudowy na czujniki aby poprawić design systemu. Zdecydowano, że cały system będzie na licencji open-source, aby każdy mógł pobrać oprogramowanie i edytować je według własnych potrzeb. Mimo, że taktowanie procesora Raspberry Pi 3 nie jest wysokie i nie poradziłby on sobie sam z zadaniami przedstawionymi na początku pracy to dzięki zastosowaniu zewnętrzych serwerów i usług na takich platformach jak Microstof Azure udało się wykonać w pełni działający system bezpieczeństwa. Przeniesiono bardzo obciążające zadania takie jak przetwarzanie obrazu na zewnętrzne platformy, które wyposażone są w znacznie bardziej zaawansowane podzespoły. Wykorzystanie natomiast usług Firebase, z którego korzystają też takie firmy jak Trivago czy Shazam pozwoliło na szybką implementację zaawansowanych funkcji takich jak uwierzytelnianie użytkowników czy aktualizację zmian w bazie danych w czasie rzeczywistym. Grupa zamierza dalej pracować nad systemem kontroli bezpieczeństwa - The Guard aby uczynić świat lepszym a przede wszystkim bezpieczniejszym miejscem do życia :)


// TODO Remove dummy doc links
\cite{MDESIGN}
\cite{RXJAVA}
\cite{KOTLIN}
\cite{RPI}
\cite{firebase}
\cite{android}
\cite{azure}
\cite{kotlin}