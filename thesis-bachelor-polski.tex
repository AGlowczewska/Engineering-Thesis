% Szkielet dla pracy inżynierskiej pisanej w języku polskim.

\documentclass[polish,bachelor,a4paper,oneside]{ppfcmthesis}


\usepackage[utf8]{inputenc}
\usepackage[T1]{fontenc}
\usepackage{textcomp}
\usepackage{listings}

% Authors of the thesis here. Separate them with \and
\author{%
Mateusz Bartos \album{122437} \and
Piotr Falkiewicz \album{122537} \and
Aleksandra Główczewska \album{122494} \and
Paweł Szudrowicz \album{122445}}
\title{System kontroli bezpieczeństwa – The Guard}                   % Note how we protect the final title phrase from breaking
\ppsupervisor{dr inż. Mariusz Nowak} % Your supervisor comes here.
\ppyear{2018}                                         % Year of final submission (not graduation!)

\begin{document}
    % Front matter starts here
    \frontmatter\pagestyle{empty}%
    \maketitle\cleardoublepage%
    % Blank info page for "karta dyplomowa"
    \thispagestyle{empty}\vspace*{\fill}%
    \begin{center}
        Tutaj przychodzi karta pracy dyplomowej;\\oryginał wstawiamy do wersji dla archiwum PP, w pozostałych kopiach wstawiamy ksero. \cite{DUMMY:1}
    \end{center}%
    \vfill\cleardoublepage%

    % Table of contents.
    \pagenumbering{Roman}\pagestyle{ppfcmthesis}%
    \tableofcontents* \cleardoublepage%

    % Main content of your thesis starts here.
    \mainmatter%
    \chapter{Wstęp}
Mimo spadku stopnia przestępczości w ostatnich latach w Polsce i stosunkowo wysokiego poziomu bezpieczeństwa zawsze lepiej mieć przy sobie narzędzia, które pozwolą kontrolować to co dzieje się w naszym domu. Kiedy jesteśmy w pracy, na wakacjach czy też zostawiliśmy nasz dom bez opieki na dłuższy czas chcielibyśmy mieć możliwość podglądu tego co się w nim dzieje i na bieżąco monitorować sytuację. W przypadku włamań zarejestrowany materiał filmowy jest bardzo cenny nie tylko dla nas ale szczególnie dla policji, która jest w stanie odtworzyć zachowania i czasami zidentyfikować sprawcę.  Bardzo często słyszy się też o zatruciach tlenkiem węgla, który nazywany jest cichym zabójcą czy wybuchach gazu w mieszkaniu. W takich chwilach człowiek zdaje sobie sprawę z zagrożenia najczęściej kiedy jest już za późno i są małe szanse aby uratować innych lokatorów naszego domu np. po zatruciu CO. 
Zbudowany system kontroli bezpieczeństwa - The Guard stara się rozwiązać te wszystkie problemy. Naszym celem było stworzenie systemu umożliwiającego analizę danych z czujników pomiarowych, monitorowanie pomieszczeń, w których zamontowano nasz system a także nagrywanie materiału video w momencie wykrycia ruchu i przechowywaniu go bezpiecznie na zewnętrznym serwerze aby był dostępny dla nas w każdym momencie i nie uległ zniszczeniu. Zadaniem systemu jest także poinformowanie o każdym niebezpieczeństwie właściciela systemu. Priorytetem był prosty i intuicyjny program obsługi, który mógłby być użyty przez każdą osobę, na każdej z najbardziej popularnych platform. Zdecydowano się na aplikację internetową i dwie aplikacje mobilne napisane natywnie dla systemu iOS i Android. Ponadto uzgodniono, że rozwiązanie będzie oparte na niezależnych modułach, które będzie można później, w łatwy sposób, zmodyfikować.

W ramach pracy przygotowano projekt całego systemu, od urządzeń zbierających dane, przez system monitorujący i analizujący zebrane dane, po aplikacje klienckie. Następnie zaimplementowano zaprojektowane wcześniej aplikacje, złożono zestawy urządzeń składających się z Raspberry Pi 3 i czujników, oraz połączono wszystkie elementy w spójny system.
Ze względu, na cel pracy oraz wykorzystane technologie i usługi, zespół oparł swoją pracę o: dokumentację usług dostępną na stronach internetowych producentów, dokumentację narzędzi dołączoną do odpowiednich repozytoriów, dokumentację sprzętu.

W ramach pracy Mateusz Bartos zrealizował aplikację mobilną przeznaczoną na system Android. Ponadto przygotował maszynę wirtualną w ramach usług oferowanych przez Microsoft Azure. Zaproponował też wykorzystane usługi Google: Firebase Realtime Database, Firebase Storage oraz Firebase Authentication. Piotr Falkiewicz wykonał projekt serwera, obsługującego zapytania aplikacji mobilnych, w oparciu o protokół HTTP oraz jest odpowiedzialny za poprawne przetwarzanie obrazu dostarczanego z urządzeń do aplikacji przy wykorzystaniu modułu Nginx Rtmp. Wprowadził do projektu obsługę usługi Firebase Storage. Paweł Szudrowicz w ramach pracy przygotował urządzenia Raspberry Pi 3 do obsługi czujników i wykonał oprogramowanie pracujące na każdym z nich. Kolejnym zrealizowanym zadaniem był projekt i wykonanie aplikacji mobilnej na urządzenia iOS. W ramach pracy Aleksandra Główczewska zaprojektowała i wykonała aplikację internetową, z wykorzystaniem języka programowania Python i frameworka Django. Ponadto jest odpowiedzialna za wprowadzenie uwierzytelniania użytkowników.


    \chapter{Architektura systemu}

Dobre zaprojektowanie architektury systemu jest fundamentalnym zadaniem. Na rynku istnieje wiele infrastruktur opartych na technologii przetwarzania w chmurze, które oferują bardzo podobne funkcjonalności. Należało wybrać te, które dawały najwięcej korzyści przy jak najniższej cenie (kosztorys omówiono w rozdziale 3). Dlatego zdecydowano się na Microsoft Azure. Dodatkowym atutem było to, że zespół miał już doświadczenie z tą usługą. Opis działania poszczególnych elementów systemu zostanie omówiony w~dalszej części pracy.

\section{Schemat}

\begin{figure}[ht] 
   \centering
   \includegraphics[width=12cm]{anton.png} 
   \caption{Schemat systemu [opracowanie własne]}
\end{figure}

Architektura została przedstawiona na schemacie (rys 2.1). Wszystkie urządzenia klienckie (iOS, Android i~Web), jak i~Raspberry Pi komunikują się z~serwerem ANTON wykupionym i~pracującym na platformie Microsoft Azure. Komunikacja pomiędzy klientami a~serwerem odbywa się dzięki REST API. Obraz natomiast przesyłany może być za pomocą dwóch protokołów RTMP i~HLS. Postanowiono, że transmisja obrazu w~systemie The Guard oparta będzie na protokole HLS ze względu na brak możliwości obsługi RTMP na iOS. Priorytetem było zapewnienie identycznych warunków i~tych samych doświadczeń użytkownika na wszystkich platformach. Było to głównym powodem całkowitego odrzucenia przesyłania obrazu przy użyciu protokołu RTMP. RTMP posiada jednak w~porównaniu do HLS jedną, aczkolwiek bardzo istotną przewagę. Jest to brak opóźnienia w~transmisji obrazu. HLS wysyła dane w~małych porcjach. Przed wysłaniem pierwszej części, konieczne jest jej nagranie. To właśnie powoduje kilkunastosekundowe opóźnienie w stosunku do RTMP, który transmituje obraz bezpośrednio. Oba protokoły omówiono dokładniej w~rozdziale 4.3.

\section{Komunikacja}

Komunikacja pomiędzy elementami systemu odbywa się na zasadach architektury REST. Wiadomości przesyłane są asynchronicznie, na wskazane wcześniej adresy.
Takie podejście gwarantuje prostotę przesyłanych komunikatów oraz skalowalność w~kontekście nowych urządzeń Raspberry strumieniujących dane oraz nowych urządzeń korzystających z~aplikacji klienckich. Początkowo projekt oparto o~zapytania GET i~POST \cite{WEBARCH}. Wprowadzenie tokenów uwierzytelniających (więcej w akapicie nt. Bezpieczeństwa) spowodowało, że wymianę komunikatów oparto tylko i~wyłącznie na zapytaniach POST. 

\paragraph{Zapytanie GET}
Metoda GET pozwala na pobranie dokumentu sieciowego, na postawie zapytania zawartego w~adresie URL. Metoda ta jest używana tylko i~wyłącznie do pobierania danych z~punktu docelowego. 

\paragraph{Zapytanie POST}
W metodzie POST, należy zamieścić wiadomość wewnątrz zapytania HTTP. Odpowiedzią na ten typ zapytania, może być zarówno kod statusu, jak i~dane, zwracane w podobnej postaci jak przy zapytaniu GET.

Wysłanie zapytania na określony adres powoduje uruchomienie specjalnej funkcji na serwerze. Każdy adres ma przypisaną osobną funkcję uruchamianą automatycznie po otrzymaniu zapytania. Funkcje te realizują operacje na bazie danych (CRUD) lub odpowiedzialne są za wysyłanie notyfikacji do wszystkich urządzeń klienta. Przykładowo (rys. 2.2) funkcja obsługująca dodanie nowego urządzenia w bazie danych, która uruchamia się po wysłaniu zapytania na adres /backend/v1/devices/add. Funkcja ta sprawdza czy zapytanie jest typu POST, pobiera przesłane dane i~wykorzystuje je do utworzenia nowego rekordu w~bazie danych.
\begin{figure}[ht] 
   \centering
   \includegraphics[width=14cm]{backend.png} 
   \caption{Funkcja obsługująca dodanie nowego urządzenia [opracowanie własne]}
\end{figure}

\section{Bezpieczeństwo}

Aplikacje wysyłając zapytania do serwera muszą potwierdzić swoją tożsamość. 
W aplikacjach mobilnych zastosowano proponowane przez Firebase rozwiązanie JSON Web Tokens. W~momencie wysłania zapytania POST do serwera, aplikacja dołącza także unikalny token generowany przez moduł Firebase Auth. Po dotarciu wiadomości do serwera, token ten weryfikowany jest przy użyciu modułu Firebase Admin SDK. Jeśli weryfikacja tokenów przebiegła pomyślnie (oba tokeny są identyczne) to udzielany jest dostęp do wykonania kodu na serwerze. W przeciwnym wypadku generowany jest błąd.

W~przypadku aplikacji internetowej zastosowano wbudowane w bibliotekę Django zabezpieczenia - przesyłanie tokenu CSRF oraz identyfikatora sesji wraz z~zapytaniem \cite{djangoCSRF}. Zabezpieczenie CSRF token uniemożliwia tzw. `Cross Site Request Forgery' tj. ataki w~których na stronie internetowej, bez wiedzy użytkownika uruchamiany jest skrypt (najczęściej w języku JavaScript). Następnie korzystając z~faktu, że użytkownik jest zalogowany strona atakująca podszywa się pod jego konto i~wysyła zapytanie do serwera, które może spowodować uruchomienie wszystkich operacji, do których upoważniony jest dany użytkownik. Aby tego uniknąć CSRF token zapisywany jest w~przeglądarce jako `ciasteczko' (eng. cookie) i dołączany do danych przesyłanych w~momenie wyboru przycisku odpowiedzialnego za przesłanie formularza. Wbudowana w~serwer Django biblioteka weryfikuje na podstawie zapisanych i~przesłanych danych sesji poprawność tokenu. W~przypadku błędu zwraca błąd serwera 403.
\\Ponieważ token przy każdym zapytaniu jest tworzony na nowo, na podstawie otwartej sesji pozostaje on rozwiązaniem przeznaczonym głównie dla aplikacji przeglądarkowych -~powyższe rozwiązanie nie byłoby komfortowe dla użytkowników aplikacji mobilnych: aplikacja musiałaby ustanowić połączenie z~serwerem (wysłać zapytanie GET na stronę główną) następnie zalogować się (wysłać zapytanie POST z danymi logowania) oraz zapisywać parę (token, id sesji) odsyłaną przez serwer. Aby ograniczyć ilość zapytań wysyłanych do serwera, w wypadku aplikacji mobilnych posłużono się inną opisaną powyżej metodą tokenów JWT.

    \chapter{Zbieranie i przetwarzanie danych z czujników}

\section*{Raspberry Pi}

Wszystkie zestawy zbudowaliśmy w oparciu o Raspberry Pi 3 v1.2. Zdecydowaliśmy się na to rozwiązanie, ponieważ bazuje on na dystrybucji Linuxa, posiada opowiednie interfejsy i złącza a także zintegrowany moduł WiFi. Minusem w stosunku do konkurencyjnego Arduino jest brak wejść analogowych. Problem rozwiązano dodając zewnętrzny przetwornik A/C.
\paragraph{Specyfikacja Raspberry Pi 3:}
\begin{itemize} 
\item Procesor 1.2 GHz
\item Liczba rdzeni 4. Quad Core
\item Pamięć RAM 1 GB
\item Pamięć Karta microSD
\item 40 GPIO
\end{itemize}
\begin{figure}[h]
	\centering
	\includegraphics[width=6cm]{raspberry.jpg}
	\caption{Raspberry Pi 3}
\end{figure}
Aby prawidłowo zainstalować oprogramowanie The Guard na dowolnym urządzeniu Raspberry Pi 3 należy wykonać poniższe czynności w terminalu:
\begin{enumerate} 
\item sudo apt-get install libx264-dev
\item cd /usr/src
\item git clone git://source.ffmpeg.org/ffmpeg.git
\item sudo ./configure --arch=armel --target-os=linux --enable-gpl --enable-libx264 --enable-nonfree
\item sudo make
\item sudo install
\item sudo nano /boot/config.txt
\item dopisać w pliku Dtoverlay=w1-gpio i Gpiopin=4
\item pip intall wiringpi
\item sudo pip install spidev
\item pip install pyrebase
\end{enumerate}
Następnym krokiem jest włączenie odpowiednich interfejsów w panelu konfiguracyjnym. Należy zmienić ustawienia zgodnie z poniższym schematem:
\begin{figure}[h]
	\centering
	\includegraphics[width=6cm]{RSettings}
	\caption{Ustawienia}
\end{figure}
W kodzie używamy biblioteki wiringpi do odczytu danych z układów cyfrowych. Należy zaznaczyć, że numeracja fizycznych pinów i numeracja pinów w bibliotece wiringPi jest różna i nie zawiera wszystkich dostępnych pinów na urządzeniu. Przykładowo odczyt pinu 1 w wiringPi jest równoznaczne z odczytem stanu na pinie 12 (GPIO18).
\begin{figure}[h]
  \centering
  \begin{minipage}[b]{0.4\textwidth}
    \includegraphics[width=\textwidth]{wiringpi.png}
    \caption{WiringPi}
  \end{minipage}
  \hfill
  \begin{minipage}[b]{0.4\textwidth}
    \includegraphics[width=\textwidth]{gpio.png}
    \caption{GPIO}
  \end{minipage}
\end{figure}
Oprogramowanie zainstalowane na Raspberry Pi odpowiedzialne jest za ciągłe monitorowanie stanów i danych z czujników pomiarowych. Po podłączeniu układu do zasilania oprogramowanie jest uruchamiane automatycznie. Pierwszą czynnością jaką wykonuje Raspberry jest wysłanie swojego numery seryjnego do bazy danych Firebase. Cały proces jest w pełni zautomatyzowany. Dzięki temu użytkownicy od razu mogą dodać urządzenie i monitorować dane z czujników na aplikacjach klienckich. Dodawanie urządzenia następuje poprzez wprowadzenie w aplikacji numery seryjnego urządzenia, które chcę dołączyć do mojego konta użytkownika.
\section*{Czujniki}

Każdy zestaw składa się z 5 czujników analogowo cyfrowych,  jednej kamery i jednego przetwornika AC. 
\paragraph{a) Specyfikacja MQ-9 - czujnik tlenku węgla:}
\begin{itemize} 
\item Zasilanie: 5 V
\item Pobór prądu: 150 mA
\item Temperatura pracy: od -10 do 50 \textdegree{}C
\item Wyjścia: analogowe oraz cyfrowe
\end{itemize}

\paragraph{b) Specyfikacja MQ-2 - czujnik LPG i dymu:}
\begin{itemize} 
\item Zasilanie: 5 V
\item Pobór prądu: 150 mA
\item Temperatura pracy: od -10 do 50 \textdegree{}C
\item Wyjścia: analogowe oraz cyfrowe
\end{itemize}

\paragraph{c) Specyfikacja czujnika wykrywania płomieni:}
\begin{itemize} 
\item Zasilanie: 3.3 V
\item Zakres wykrywanej fali: 760 do 1100nm
\item Kąt detekcji: od 0 do 60 stopni
\item Temperatura pracy: od -25 do 85 \textdegree{}C
\end{itemize}

\paragraph{d) Specyfikacja DS18B20 - czujnik temperatury:}
\begin{itemize} 
\item Zasilanie: 3.3 V
\item Zakres pomiarowy: od -55 do 125 \textdegree{}C
\end{itemize}

\paragraph{e) Kamera:}
\begin{itemize} 
\item Wykorzystano moduł kamery Raspberry Pi element14
\item Kamera 5MP - wspierająca nagrywanie 30 klatek na sekundę w rozdzielczości Full HD
\end{itemize}

\paragraph{f) Specyfikacja MCP3008 - przetwornik A/C:}
\begin{itemize} 
\item Zasilanie: od 2.7V do 5.5V
\item Pobór prądu: 0.5 mA
\item Interfejs komunikacyjny: SPI
\item Liczba kanałów: 8
\item Rozdzielczość: 10bit
\item Czas konwersji: 10us
\end{itemize}

\begin{figure}[h]
	\centering
	\includegraphics[width=15cm]{GuardSchem}
	\caption{Schemat układu The Guard}
\end{figure}

Niestety żaden model Raspberry nie posiada wbudowanego przetwornika analogowo cyfrowego dlatego konieczne było użycie układu zewnętrzenego. Wybraliśmy przetwornik MCP3008 ze względu na jego nisko koszt i interfejs SPI, który jest wspierany przez Raspberry Pi.
MCP3008 to 10-bitowy przetwornik analogowy cyfrowy. Zasilany jest napięciem 5V, napięcie VRef = 5V.  Skoro jest to przetwornik 10-bitowy jest w stanie wykryć 1024 stanów. Wykrywana przez niego minimalna różnica napięć na wejściu wynosi 
\begin{equation}
1 * 5V / 1024 = 4.88mV
\end{equation}
Posiada 8 kanałów jednak w projekcie wykorzystano tylko 2 – do podłączenia czujników MQ-9 i MQ-2.

\paragraph{Interfejs SPI:}
\begin{figure}[h]
	\centering
	\includegraphics[width=6cm]{SPI.png}
	\caption{Interfejs SPI}
\end{figure}
SPI jest to interfejs synchroniczny. Może być do niego podłączone wiele urządzeń typu slave, jednak tylko z jednym urządzeniem Master, który generuje zegar. Master poprzez linię SS wybiera urządzenie z którym chce się komunikować.  \\
Interfejs ten zawiera jeszcze 3 linie:
\begin{enumerate} 
\item MOSI (ang. Master Output Slave Input): \\
Poprzez tę linię wysyłane są dane z Raspberry Pi do przetwornika analogowo cyfrowego MCP3008.
\item MISO (ang. Master Input Slave Output):\\
Poprzez tę linię wysyłane są dane z przetwornika AC do układu Master czyli w naszym przypadku Raspberry Pi 3
\item SCLK (ang. Serial CLocK) :\\
Ta linia wykorzystywana jest do przesłania zegara wygenerowanego z Rapberry Pi 3
\end{enumerate}
Do komunikacji poprzez ten interfejs wykorzystano bibliotekę spiDev. \\


Każdy układ monitoruje wskaźniki pomiarowe z czujników analogowych i cyfrowych. W przypadku wykrycia wskazań, które w znaczący sposób odbiegają od normy informuje właściciela o zagrożeniu. Informacja ta wysyłana jest do wszystkich urządzeń, które posiada właściciel.  Analizując dane z czujników analogowych w czytym powietrzu, które wynoszą wtedy odpowiednio:\\
Czujnik MQ-9: 0.15 – 0.2\\
Czujnik MQ-2: 0.05 – 0.15\\
Przyjęto, że granicą wysłania notyfikacji do urządzeń użytkownika jest przekroczenie progu 0.3. Wartości te to znormalizowane dane z układu przetwornika AC, który jak już wcześniej wspomniałem wykrywa 1024 stany. Odczytywane wartości bezpośrednio na wyjściu cyfrowym przetwornika MCP3008 dla czujnika MQ-9 w czystym powietrzu to około 170. Stąd 170/1024 = 0.166. Wysłanie notyfikacji wiąże się z otrzymaniem wartości min. 308 bezpośrednio na wyjściu cyfrowym.
Reszta to czujniki cyfrowe, które informują m.in. o wykryciu ognia i wykryciu ruchu. W przypadku detekcji zagrożenia wysyłają one na wyjście stan niski i utrzymują go przez kilka sekund. W kodzie jednak wykonujemy instrukcje negacji tak, aby stan wysoki informował o niebezpieczeństwie a stan nisko reprezentował jego brak. Na obu czujnikach znajduje się potencjometr, za pomocą którego dowolnie można ustawiać czułość czujnika.
Bezpośredni odczyt danych z czujników następuje nieprzerwanie co 2 sekundy. Nie należy obawiać się, że czujnik ruchu nie wykryje zagrożenia gdyż nie zostanie odczytany w poprawnym czasie, ponieważ utrzymuje on stan wysoki przez 5 sekund po wykryciu ruchu.
Oprogramowanie oprócz monitorowania danych z czujników wysyła je także do bazy danych Firebase. Dzięki temu mamy podgląd wszystkich danych z każdego z pomieszczeń w czasie rzeczywistym. Dodatkowo w przypadku wykrycia zagrożenia czyli przekroczeniu progu o którym mowa wyżej wysyłamy push notyfikacje do wszystkich urządzeń użytkownika i zapisujemy wysłaną notyfikację w bazie danych Django. Dzięki zapisywaniu danych jesteśmy w stanie odtworzyć całą historię wydarzeń w systemie.
Aby zapewnić wydajny i pewny system bezpieczeństwa przy otrzymaniu wysokich wartości na czujnikach i wysłaniu notyfikacji zapisujemy timestamp zdarzenia. Dzięki temu nie ma możliwości zasypania właściciela miliardem informacji gdyż każda kolejna notyfikacja zostanie wysłana po min. 10 min od poprzedniej przy założeniu, że nadal stan na czujniku jest wysoki.




















\section*{Obsługa wideo}

Opis procesu, opis Nginxa, opis RTMP i HLS
    \chapter{Rozwiązania chmurowe}

\section*{Microsoft Azure}

Aby zapewnić wysoki poziom bezpieczeństwa oraz dostępności systemu zdecydowano się na skorzystanie z chmury Micrsoft Azure.

\section*{Aplikacja serwerowa}

Opis Django

\section*{Baza danych}

Opis technologii, schematy tabel
    \chapter{Aplikacje klienckie}

\section*{Aplikacja Android}

Opis Androida, screenshoty


\section*{Aplikacja iOS}
Aplikacja przeznaczona jest na urządzenia z systemem operacyjnym iOS od wersji 10.0. 
Nie wspiera ona wcześniejszych wersji ze względu na nowe funkcje, które Apple wprowadziło wraz z pojawieniem się iOS 10.0 (m.in. klasa UNUserNotificationCenter). Jednak jak wynika z wykresu niżej (numer rysunku) 92\% wszystkich obecnych użytkowników tego systemu jest w stanie zainstalować aplikację a liczba ta stale rośnie. Aplikacja wykonana została wspierając zarówno telefony komórkowe iPhone jak i tablety iPad. 
\begin{figure}[h]
	\centering
	\includegraphics[width=6cm]{iOSstat}
	\caption{Udziały wersji systemu iOS w rynku}
\end{figure}
Napisana jest w stosunkowo nowym języku Swift (został zaprezentowany przez Apple w 2014r na konferencji WWDC) w oparciu o architekturę MVC (Model-View-Controller) wykorzystując przy programowanie reaktywne i funkcjonalne. 
Programowanie reaktywne zrealizowano przy pomocy biblioteki RxSwift. Wykorzystano je m.in w celu wznowienia streamu obrazu z kamery w momencie przejścia aplikacji z trybu Background do trybu Foreground. Oznacza to, że w momencie wyjścia z aplikacji ale pozostawiając ja działającą w tle i po chwili uruchomienia jej ponownie traciliśmy obraz Video, ze względu na politykę bezpieczeństwa Apple, która nie zezwala aby aplikacje pobierały dane przez dłuższy okres czasu kiedy aplikacja pracuje w tle. Dzięki programowaniu reaktywnemu problem ten został rozwiązany co prezentuje poniższy kod:
\begin{verbatim}
let appDelegate = UIApplication.shared.delegate as! AppDelegate
        appDelegate.inBackground.asObservable().subscribe(onNext: { (value) in
            if let streamView = self.streamView {
                if let player = self.currentPlayer {
                    if value == false {
                        self.streamVideoFrom(urlString: self.currentUrlString!)
                        print("Enter foreground")
                    } else {
                        print("Enter background")
                        streamView.layer.sublayers?.forEach({ (layer) in
                            layer.removeFromSuperlayer()
                        })
                    }
                }
            }
        }).disposed(by: disposeBag)
\end{verbatim}
Zmienna inBackground ustawiana jest oddzielnej klasie AppDelegate na wartość true kiedy aplikacja przechodzi w tryb background i na wartość false w przeciwnym wypadku. Kod powyżej uruchamia się za każdym razem przy zmianie tej wartości i uruchamia stream po każdym ponownym uruchomieniu programu.
Programowanie funkcjonalne natomiast wykorzystane jest w miejscach gdzie konieczne jest przekształcenie danych do innej postaci:
\begin{verbatim}
lastNotification = notifications.array.sorted(by: { (n1, n2) -> Bool in
	n1.date > n2.date 
}).filter({ (notif) -> Bool in return notif.type == "PIRSensor"}).first
\end{verbatim}
Na tablicy z notyfikacjami zastosowano szereg funkcji: posortowano je malejąco według daty, przefiltrowano w taki sposób aby wybrać tylko te o typie 'PIRSensor' czyli tylko te pochodzące z czujnika ruchu. Na sam koniec wybrano tylko jeden pierwszy element z wybranych i wynik wpisano do zmiennej lastNotification.


Instalacja zewnętrznych bibliotek odbywa się za pomocą CocoaPods. Jest to menadżer zależności dzięki któremu szybko możemy wyszukać i zainstalować wymagane oprogramowanie. Wszystkie użyte zależności przedstawia poniższy listing: 

\begin{verbatim}
  pod 'Moya'
  pod 'MBProgressHUD', '~> 1.0'
  pod 'RxSwift',    '~> 4.0'
  pod 'RxCocoa',    '~> 4.0'
  pod 'IHKeyboardAvoiding'
  pod 'Moya-SwiftyJSONMapper'
  pod 'Firebase/Core'
  pod 'Firebase/Messaging'
  pod 'Firebase/Auth'
  pod 'Firebase/Database'
  pod 'M13ProgressSuite'
\end{verbatim}
Moya używana jest do asynchronicznej REST-owej komunikacji z serwerem Django. SwiftyJSONMapper przydatna okazuje się do przekształcenia odpowiedzi serwera w postaci JSON'a do wcześniej zdefiniowanego modelu. MBProgressHUD umożliwia wyświetlanie ekranu ładowania danych podczas pobierania informacji z serwera. RxSwift i RxCocoa to biblioteki do programowania reaktywnego. Moduły Firebas/Core itp. służą do komunikacji z serwerami Firebase. Ostatni 'pod M13ProgressSuite' służy do rysowania wykresów i animowanych elementów graficznych w systemie iOS.

Po uruchomieniu aplikacji pierwszym widokiem jest ekran logowania i rejestracji użytkowników(rys 5.2). 
\begin{figure}[h]
	\centering
	\includegraphics[width=6cm]{login.png}
	\caption{Ekran logowania}
\end{figure}
Po prawidłowej autoryzacji danych użytkownika wprowadzonych podczas logowania ukaże się nam główny widok aplikacji. W górnej części mamy do wyboru 5 sekcji:
sekcja czujników, sekcja historii notyfikacji, sekcja zagrożeń przy wykryciu ruchu, sekcja streamu na żywo, sekcja ustawień. Wszystkie te sekcje dotyczą konkretnego urządzenia wybranego w pasku na dole ekranu. Przy pierwszym uruchomieniu nie będziemy posiadali żadnych urządzeń przypisanych do naszego konta użytkownika. Aby dodać pierwsze i kolejne stacje, od których chcemy otrzymywać notyfikacje o zagrożeniach a także śledzić i monitorować informacje z czujników należy wybrać przycisk "New" z plusikiem w dolnej części ekranu. Pojawi się okno z prośbą o wpisanie numeru identyfikującego jednoznacznie urządzenie. Po chwili dodany Guard będzie widoczny w na liście.

\paragraph{Sekcja czujników:}
Jest to jedna z najważniejszych sekcji aplikacji (rys 5.3).  Otrzymuje ona dane z czujników w czasie rzeczywistym i prezentuje je użytkownikowi.  W zależności od koloru prezentowanej wartości z czujnika użytkownik analizuje zagrożenie. Kolor zielony reprezentuje bezpieczne i prawidłowe odczyty na czujnikach, kolor pomarańczowy średnie, kolor czerwony reprezentuje bardzo wysoki poziom niebezpieczeństwa. Implementacja tej funkcjonalności zrealizowana została przy pomocy modelu HSV a nie RGB, dzięki temu zmieniając parametr Hue zmieniamy barwę przy stałym nasyceniu i jasności. Wartość tego parametru równa 120\textdegree{} odpowiada kolorowi zielonemu, kolor czerwony to 0\textdegree{}. Przekształcając wartość otrzymaną z czujników, która jest z zakresu [0-1] na wartość z przedziału [120-0] otrzymano wyżej wspomniany efekt. 
Poniżej przedstawiono fragment konwersji danych z czujników na kolor w modelu HSV, gdzie zmienna sensors[0] reprezentuje czujnik LPG.
\begin{verbatim}
UIColor(hue: CGFloat(0.33 - (sensors[0].value * 0.33)),
saturation: 1, brightness: 1, alpha: 1)
\end{verbatim}

\begin{figure}[h]
	\centering
	\includegraphics[width=6cm]{sensors.png}
	\caption{Sekcja czujników}
\end{figure}

\paragraph{Sekcja historii notyfikacji:}
W tej sekcji użytkownik ma dostęp do historii zdarzeń w systemie (rys 5.4). Po zaznaczeniu interesującego nas daty reprezentującej moment wystąpienia zagrożenia i wybranu przycisku 'preview' prezentowana jest informacja o miejscu niebezpieczeństwa i jego rodzaju. 

\begin{figure}[h]
	\centering
	\includegraphics[width=6cm]{history.png}
	\caption{Sekcja historii zdarzeń}
\end{figure}


\paragraph{Sekcja ustawień:}
Ustawienia dotyczące zaznaczonego na dole ekranu urządzenia (rys 5.5). Użytkownik ma możliwość zmiany nazwy urządzenia, które zazwyczaj reprezentuje miejsce, w którym znajduje się stacja pomiarowa. Istnieje również możliwość uzbrojenia i wyłączenia konkretnego czujnika. Sprowadza się to do tego, że w przypadku zaznaczenia opcji "Disarmed" użytkownik nie będzie otrzymywał kolejnych notyfikacji o zagrożeniach. Opcja ta może okazać się przydatna w momencie uszkodzenia któregoś z modułów i tym samym błednych danych wysyłanych z czujników.

\begin{figure}[h]
	\centering
	\includegraphics[width=6cm]{settings.png}
	\caption{Sekcja historii zdarzeń}
\end{figure}








\section*{Aplikacja internetowa}

Opis weba, screenshoty
    \input{06-zakonczenie.tex}

    % All appendices and extra material, if you have any.
    \cleardoublepage\appendix%
    \input{0a-zalacznik.tex}
    \input{0b-pisanie-w-latexu.tex}

    % Bibliography (books, articles) starts here.
    \bibliography{bibliography} 
    \bibliographystyle{plain}
  %  {\raggedright\sloppy\small\bibliography{bibliography}}

    % Colophon is a place where you should let others know about copyrights etc.
    \ppcolophon

\end{document}
